\section{Introduction}

The current DNS relies heavily on its
thirteen root servers, which are responsbile for providing
information about the root of the hierarchy and many top-level domains.
The Internet Software Consortium, which runs one of the root servers,
reported a load of 272 million queries per day in 2001~\cite{isc-root}.
Chord balances load as a side effect of its lookup and storage 
algorithms, thus avoiding such hot spots.

We propose a distributed domain name service, DDNS,
using a robust peer-to-peer lookup service like Chord.
Such a system avoids some of the most significant
problems with the current DNS implementation.
 
Leslie Lamport's supposed characterization of a
distributed system (as one in which you can't get work
done because of a machine you have never heard of is down)
is all too true in the current implementation of {DNS}.
Jung {\it et al.}~\cite{dnscache:sigcommimw01} found that approximately
35\% of DNS queries never receive an answer or receive
an incorrect negative answer.
Many of these failures are due to improperly configured
root DNS entries (name server, or NS records).
Such failures can go undetected because of the hierarchical
nature of DNS: lookups within an administrative domain might
succeed but that domain is unknowningly cut off from the rest
of the hierarchy, so that lookups from outside will fail.
Using Chord avoids this problem: since there is no hierarchical
caching, lookups behave identically regardless of source.

**How to control caching on others name servers...**

Distributing DNS data in a peer-to-peer network does have some
disadvantages.In particular, we cannot easily dynamically 
generate DNS responses. We discuss various solutions to 
this problem in the implementation section.



