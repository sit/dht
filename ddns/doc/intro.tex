\section{Introduction and Related Work}

In the beginnings of the internet, host names were kept
in a centrally-administered text file, {\tt hosts.txt}, that
all sites mirrored from the SRI Network Information Center.
By the early 1980s the host database had become too large
to disseminate in a cost-effective manner.
In response, Mockapetris (et al?XXX) began the design and
implementation of a distributed database that we now know
as the internet domain name system (DNS).
DNS started to replace {\tt hosts.txt} in 1985 and has remained
substantially the same since then~\ref{dns-concept:rfc, dns}.

Looking back at DNS in 1988, Mockapetris and Dunlap 
listed what they believed to be the surprises, successes,
and shortcomings of the system.  Of the six successes,
three (variable depth hierarchy, organizational structure
of names, and mail address cooperation) relate directly 
to the adoption of an administrative hierarchy for the names~\ref{dns}.
The administrative hierarchy of DNS is reflected in the 
service of the database: in typical usage, an
entity is responsible not only for information about its hosts
but also for the service of that information.

In fact, the service structure {\em needed} to reflect the 
administrative hierarchy: this provided a modicum of authentication
for the returned data.
Because servers were effectively authenticated by return address,
the system was susceptible to DNS server impersonation by 
IP spoofing or other means.
In response to concerns about this and
other attacks, the DNS security extensions (DNSSEC) were
developed in the late 1990s.
DNSSEC provides a mechanism for clients to verify that the
records they retreive are authentic~\ref{dnssec}.
Although DNSSEC is not yet in widespread use, its specification
has stabilized, and it is supported by a variety of name server
implementations.

Due to the advent of DNSSEC, the service structure no longer needs
to reflect the administrative hierarchy.
This observation enables the exploration of alternate service 
structures to achieve desirable properties not possible
with conventional DNS.  In this paper we explore one such
alternate service structure: we store and retrieve
DNS records using Chord~\ref{chord:sigcomm}, a peer-to-peer
lookup service.

