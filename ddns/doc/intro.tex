\section{Introduction and Related Work}

In the beginnings of the Internet, host names were kept
in a centrally-administered text file, {\tt hosts.txt}, that
included all sites mirrored from the SRI Network Information Center.
By the early 1980s the host database had become too large
to disseminate in a cost-effective manner.
In response, Mockapetris {\it et al.} began the design and
implementation of a distributed database that we now know
as the internet domain name system (DNS).
DNS started to replace {\tt hosts.txt} in 1985 and has remained
substantially the same since then~\cite{dns-concept:rfc, dns}.

Looking back at DNS in 1988, Mockapetris and Dunlap~\cite{dns}
listed what they believed to be the surprises, successes,
and shortcomings of the system.  Of the six successes,
three (variable depth hierarchy, organizational structure
of names, and mail address cooperation) relate directly 
to the adoption of an administrative hierarchy for the names.
The administrative hierarchy of DNS is reflected in the 
service of the database: in typical usage, an
entity is responsible not only for information about its hosts
but also for the service of that information.

In fact, the service structure {\em needed} to reflect the 
administrative hierarchy: this provided a modicum of authentication
for the returned data.
Because servers were effectively authenticated by return address,
a system was susceptible to DNS server impersonation by 
IP spoofing or other means.
In response to concerns about this and
other attacks, the DNS Security Extensions~\cite{dnssec:rfc}
 (DNSSEC) were developed in the late 1990s.
DNSSEC provides a mechanism for clients to verify that the
records they retreive are authentic.
Although DNSSEC is not yet in widespread use, its specification
has stabilized, and it is supported by a variety of name server
implementations (eg. ISC BIND server software). 

DNS performance studies have revealed some problems with the
current DNS. In a study by Danzig {\it et al.}~\cite{dnsroot:sigcomm92},
most DNS traffic was caused by misconfiguration and faulty implementation
in the name servers. It also found that one third of the 
traffic that traversed the NSFNet was directed to one of 
the seven root name servers (in 1992). 
Jung {\it et al.}~\cite{dnscache:sigcommimw01}
found that approximately 35\% of DNS queries never receive
an answer or receive a negative answer, and attributed
many of these failures to 
improperly configured name servers or incorrect name server (NS) records.
The study (in 2000) also reported as much as 18\% of DNS lookups 
reaching the root servers. 
CERT~\cite{cert} issued warnings on recent 
denial-of-service attacks on name 
servers due to vulnerabilities in name server implementations.

These reports have provoked us to rethink the current 
implementation of the DNS. 
Due to the advent of DNSSEC, the service structure no longer needs
to reflect administrative hierarchy. 
This observation enables the exploration of alternate service 
structures to achieve desirable properties not possible
with conventional DNS.  In this paper, we explore one such
alternate service structure, DDNS--a system that stores and retrieves
DNS records using Chord~\cite{chord:sigcomm}, a peer-to-peer
lookup service.

