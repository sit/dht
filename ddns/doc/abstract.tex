\begin{abstract}

The traditional Domain Name Service (DNS) couples ownership
of domains with the responsibility of servicing them.
The result is lack of flexibility in designing the name
resolution system. We propose an alternative system, DDNS,
which separates authority and service in DNS using 
Chord~\cite{chord:sigcomm}, a peer-to-peer
lookup service. DDNS replaces the name server hierarchy in
DNS, thus XXXeliminating some administrative problemsXXX,
and improving fault-tolerance in the DNS.
Experiments XXX show XXX that DDNS achieves good load balancing
and comparable latency to the current DNS, and
is resilient to popular name lookups.
We also discuss the pros and cons that arise 
when divorcing DNS data away from the hosts 
that are responsible for them.

%is sensitive to 
%server availability and does a poor job of load balancing. 
%We propose DDNS, a distributed DNS using
%Chord~\cite{chord:sigcomm}, a peer-to-peer lookup service.
%Due to Chord's fault tolerant properties, our system is more resilient to 
%failures and does better load balancing than traditional DNS.
%We plan to verify that DDNS gives reasonable lookup latency.
%We also discuss some interesting complications that arise
%when divorcing DNS data from the hosts that are responsible for it.
\end{abstract}

